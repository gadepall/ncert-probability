\begin{enumerate}[label=\thesection.\arabic*,ref=\thesection.\theenumi]
\item  The random variable $X$ has a probability distribution \pr{X} of the following form, where $k$ is some number 
\begin{align}
  \pr{X} =
    \begin{cases}
      k,  & x=0\\
      2k, & x=1\\
      3k, & x=2\\
      0 , & \text{otherwise}
    \end{cases}       
\end{align}
		\begin{enumerate}
			\item
 Determine the value of $k$ 

\item  Find \pr{X < 2},\pr{X \leq 2},\pr{X \geq 2}  
		\end{enumerate}
\solution
%The desired probabilities are 
% 
\begin{enumerate} 
\item $X \sim \bnm{2}{\frac{1}{2}}$
 \begin{align}
	 p_{X}(2) = \comb{2}{2}\frac{1}{2}^{2} = \frac{1}{4}
\end{align}
\item $X \sim \bnm{3}{\frac{1}{2}}$
 \begin{align}
	 p_{X}(3) = \comb{2}{0}\frac{1}{2}^{3} = \frac{1}{8}
\end{align}
\item $X \sim \bnm{4}{\frac{1}{2}}$
 \begin{align}
	 p_{X}(4) = \comb{4}{4}\frac{1}{2}^{4} = \frac{1}{16}
\end{align}
\end{enumerate}

\item State which of the following are not the probability distributions of a random 
variable. Give reasons for your answer
\renewcommand{\labelenumii}{\roman{enumii}}
\begin{enumerate}

\item \begin{table}[ht!]\centering
%\input{ncert/13/4/tables/Book.tex}
\end{table}

\item \begin{table}[ht!]\centering
%\input{ncert/13/4/tables/Book2.tex}
\end{table}

\item  \begin{table}[ht!]\centering
%\input{ncert/13/4/tables/Book3.tex}	
\end{table}

\item  \begin{table}[ht!]\centering
%\input{ncert/13/4/tables/Book5.tex}	
\end{table} 


\end{enumerate}
\item A random variable X has the following probability distribution\\

Determine

\begin{enumerate}
\begin{table}[ht!]\centering
%\input{ncert/13/4/tables/Book10.tex}
\end{table}
\item k
\item P$(X < 3)$
\item P$(X > 6)$
\item P$(0 < X < 3)$

\end{enumerate}

\item The random variable X has a probability distribution P(X) of the following form,
where k is some number :
\[P(x)=\begin{cases}
k, & \mbox{if}\; x= 0\\
2k, & \mbox{if}\; x= 1\\
3k, & \mbox{if}\; x= 2\\
0, & otherwise
\end{cases}\]
\begin{enumerate}
\item Determine the value of k.
\item Find P $(X < 2)$, P $(X \leq 2)$, P$(X \geq 2)$
\end{enumerate}
\item
A game consists of spinning an arrow which comes to rest pointing at one of the regions (1, 2 or 3) (Fig. 13.1). Are the outcomes 1, 2 and 3 equally likely to occur? Give reasons.\\
\begin{figure}[!ht]
	\begin{center}
		
		\resizebox{\columnwidth}{!}{\input{exemplar/10/13/2/6/figs/circle.tex}}
	\end{center}
	\caption{Fig.13.1}
	\label{fig:circle.tex}	
\end{figure}\\
\solution
%\begin{table}[H]
	\centering
\begin{tabular}{|c|c|c|}
\hline
Random variable &Value &Definition\\ \hline
\multirow{3}{*}{X} &0 &Slips of Rs 1\\
&1 &Slips of Rs 5\\
&2 &Slips of Rs 13\\ \hline
\multirow{2}{*}{Y} &0 &Box A\\
&1 &Box B\\\hline
\end{tabular}
\caption{}
\label{tab:Distribution}
\end{table}
See \tabref{tab:Distribution}.
\begin{align}
p_{Y}\brak{k}= \begin{cases} 
      \frac{1}{3} & {k=0} \\
      \frac{2}{3 }& {k=1} 
   \end{cases}
   \\
p_{Y|X}\brak{0|0} = \frac{19}{25}\, 
p_{Y|X}\brak{0|1} = \frac{6}{25}\,
p_{Y|X}\brak{1|0} = \frac{45}{50}\,
p_{Y|X}\brak{1|2} = \frac{5}{50}
\end{align}
The desired probability is the probability that a slip drawn at random is marked other than Rs 1,
\begin{align}
&=1-p_X\brak{0}\\
&= p_X(1) + p_X(2)
\end{align}
Using Bayes theorem,
\begin{align}
&= p_Y\brak{0} \times \pr{Y=0 | X=1} + p_Y\brak{1} \times \pr{Y=1|X=2}\\
&=\frac{1}{3} \times \frac{6}{25} + \frac{2}{3} \times \frac{5}{50}\\
&=\frac{11}{75}
\end{align}

\newpage

%\tableofcontents

\bigskip

\renewcommand{\thefigure}{\theenumi}
\renewcommand{\thetable}{\theenumi}
%\renewcommand{\theequation}{\theenumi}

%\begin{abstract}
%%\boldmath
%In this letter, an algorithm for evaluating the exact analytical bit error rate  (BER)  for the piecewise linear (PL) combiner for  multiple relays is presented. Previous results were available only for upto three relays. The algorithm is unique in the sense that  the actual mathematical expressions, that are prohibitively large, need not be explicitly obtained. The diversity gain due to multiple relays is shown through plots of the analytical BER, well supported by simulations. 
%
%\end{abstract}
% IEEEtran.cls defaults to using nonbold math in the Abstract.
% This preserves the distinction between vectors and scalars. However,
% if the journal you are submitting to favors bold math in the abstract,
% then you can use LaTeX's standard command \boldmath at the very start
% of the abstract to achieve this. Many IEEE journals frown on math
% in the abstract anyway.

% Note that keywords are not normally used for peerreview papers.
%\begin{IEEEkeywords}
%Cooperative diversity, decode and forward, piecewise linear
%\end{IEEEkeywords}



% For peer review papers, you can put extra information on the cover
% page as needed:
% \ifCLASSOPTIONpeerreview
% \begin{center} \bfseries EDICS Category: 3-BBND \end{center}
% \fi
%
% For peerreview papers, this IEEEtran command inserts a page break and
% creates the second title. It will be ignored for other modes.
%\IEEEpeerreviewmaketitle




\item Apoorv throws two dice once and computes the product of the numbers appearing on the dice. Peehu throws one die and squares the number that appears on it. Who has the better chance of getting the number 36? Why?\\
\solution
%\begin{table}[!ht]
\input{exemplar/12/13/3/41/tables/table.tex}
\caption{random variables of objects}
\label{tab:exemplar 12.13.3.41}
\end{table}
\begin{align}
\pr{X=i}=
\begin{cases}
\frac{1}{6},\text{when i=1}\\
\frac{2}{6},\text{when i=2}\\
\frac{3}{6},\text{when i=3}
\end{cases}
\end{align}

we know that that the conditional probability is defined as

                     $\pr{A|B}=\frac{\pr{A,B}}{\pr{B}}$


\begin{enumerate}
\item
The probability that a red ball will be selected is:
\begin{align}
\pr{Y=1}&=\pr{Y=1,X=1}+\pr{Y=1,X=2}+\pr{Y=1,X=3}\\
&=\pr{X=1}\times\pr{Y=1|X=1}+\pr{X=2}\times\pr{Y=1|X=2}+\pr{X=3}\times\pr{Y=1|X=3}\\
&=\frac{1}{6}\times\frac{3}{3}+\frac{2}{6}\times\frac{2}{3}+\frac{3}{6}\times0\\
&=\frac{7}{18}
\end{align}
\item
The probability that a white ball will be selected is:
\begin{align}
\pr{Y=0}&=\pr{Y=0,X=1}+\pr{Y=0,X=2}+\pr{Y=0,X=3}\\
&=\pr{X=1}\times\pr{Y=0|X=1}+\pr{X=2}\times\pr{Y=0|X=2}+\pr{X=3}\times\pr{Y=0|X=3}\\
&=\frac{1}{6}\times0+\frac{2}{6}\times\frac{1}{3}+\frac{3}{6}\times\frac{3}{3}\\
&=\frac{11}{18}
\end{align}
\end{enumerate}




\item 6 boys and 6 girls sit in a row at random. The probability that all the girls sit
together is
\begin{enumerate}
	\item $\frac{1}{432}$
	\item $\frac{12}{431}$
	\item $\frac{1}{132}$
	\item none of the above 
\end{enumerate}
			%\begin{table}[H]
	\centering
\begin{tabular}{|c|c|c|}
\hline
Random variable &Value &Definition\\ \hline
\multirow{3}{*}{X} &0 &Slips of Rs 1\\
&1 &Slips of Rs 5\\
&2 &Slips of Rs 13\\ \hline
\multirow{2}{*}{Y} &0 &Box A\\
&1 &Box B\\\hline
\end{tabular}
\caption{}
\label{tab:Distribution}
\end{table}
See \tabref{tab:Distribution}.
\begin{align}
p_{Y}\brak{k}= \begin{cases} 
      \frac{1}{3} & {k=0} \\
      \frac{2}{3 }& {k=1} 
   \end{cases}
   \\
p_{Y|X}\brak{0|0} = \frac{19}{25}\, 
p_{Y|X}\brak{0|1} = \frac{6}{25}\,
p_{Y|X}\brak{1|0} = \frac{45}{50}\,
p_{Y|X}\brak{1|2} = \frac{5}{50}
\end{align}
The desired probability is the probability that a slip drawn at random is marked other than Rs 1,
\begin{align}
&=1-p_X\brak{0}\\
&= p_X(1) + p_X(2)
\end{align}
Using Bayes theorem,
\begin{align}
&= p_Y\brak{0} \times \pr{Y=0 | X=1} + p_Y\brak{1} \times \pr{Y=1|X=2}\\
&=\frac{1}{3} \times \frac{6}{25} + \frac{2}{3} \times \frac{5}{50}\\
&=\frac{11}{75}
\end{align}

\newpage

%\tableofcontents

\bigskip

\renewcommand{\thefigure}{\theenumi}
\renewcommand{\thetable}{\theenumi}
%\renewcommand{\theequation}{\theenumi}

%\begin{abstract}
%%\boldmath
%In this letter, an algorithm for evaluating the exact analytical bit error rate  (BER)  for the piecewise linear (PL) combiner for  multiple relays is presented. Previous results were available only for upto three relays. The algorithm is unique in the sense that  the actual mathematical expressions, that are prohibitively large, need not be explicitly obtained. The diversity gain due to multiple relays is shown through plots of the analytical BER, well supported by simulations. 
%
%\end{abstract}
% IEEEtran.cls defaults to using nonbold math in the Abstract.
% This preserves the distinction between vectors and scalars. However,
% if the journal you are submitting to favors bold math in the abstract,
% then you can use LaTeX's standard command \boldmath at the very start
% of the abstract to achieve this. Many IEEE journals frown on math
% in the abstract anyway.

% Note that keywords are not normally used for peerreview papers.
%\begin{IEEEkeywords}
%Cooperative diversity, decode and forward, piecewise linear
%\end{IEEEkeywords}



% For peer review papers, you can put extra information on the cover
% page as needed:
% \ifCLASSOPTIONpeerreview
% \begin{center} \bfseries EDICS Category: 3-BBND \end{center}
% \fi
%
% For peerreview papers, this IEEEtran command inserts a page break and
% creates the second title. It will be ignored for other modes.
%\IEEEpeerreviewmaketitle




\item A card is selected from a deck of 52 cards. The probability of its being a red face card is
%\begin{table}[H]
	\centering
\begin{tabular}{|c|c|c|}
\hline
Random variable &Value &Definition\\ \hline
\multirow{3}{*}{X} &0 &Slips of Rs 1\\
&1 &Slips of Rs 5\\
&2 &Slips of Rs 13\\ \hline
\multirow{2}{*}{Y} &0 &Box A\\
&1 &Box B\\\hline
\end{tabular}
\caption{}
\label{tab:Distribution}
\end{table}
See \tabref{tab:Distribution}.
\begin{align}
p_{Y}\brak{k}= \begin{cases} 
      \frac{1}{3} & {k=0} \\
      \frac{2}{3 }& {k=1} 
   \end{cases}
   \\
p_{Y|X}\brak{0|0} = \frac{19}{25}\, 
p_{Y|X}\brak{0|1} = \frac{6}{25}\,
p_{Y|X}\brak{1|0} = \frac{45}{50}\,
p_{Y|X}\brak{1|2} = \frac{5}{50}
\end{align}
The desired probability is the probability that a slip drawn at random is marked other than Rs 1,
\begin{align}
&=1-p_X\brak{0}\\
&= p_X(1) + p_X(2)
\end{align}
Using Bayes theorem,
\begin{align}
&= p_Y\brak{0} \times \pr{Y=0 | X=1} + p_Y\brak{1} \times \pr{Y=1|X=2}\\
&=\frac{1}{3} \times \frac{6}{25} + \frac{2}{3} \times \frac{5}{50}\\
&=\frac{11}{75}
\end{align}

\newpage

%\tableofcontents

\bigskip

\renewcommand{\thefigure}{\theenumi}
\renewcommand{\thetable}{\theenumi}
%\renewcommand{\theequation}{\theenumi}

%\begin{abstract}
%%\boldmath
%In this letter, an algorithm for evaluating the exact analytical bit error rate  (BER)  for the piecewise linear (PL) combiner for  multiple relays is presented. Previous results were available only for upto three relays. The algorithm is unique in the sense that  the actual mathematical expressions, that are prohibitively large, need not be explicitly obtained. The diversity gain due to multiple relays is shown through plots of the analytical BER, well supported by simulations. 
%
%\end{abstract}
% IEEEtran.cls defaults to using nonbold math in the Abstract.
% This preserves the distinction between vectors and scalars. However,
% if the journal you are submitting to favors bold math in the abstract,
% then you can use LaTeX's standard command \boldmath at the very start
% of the abstract to achieve this. Many IEEE journals frown on math
% in the abstract anyway.

% Note that keywords are not normally used for peerreview papers.
%\begin{IEEEkeywords}
%Cooperative diversity, decode and forward, piecewise linear
%\end{IEEEkeywords}



% For peer review papers, you can put extra information on the cover
% page as needed:
% \ifCLASSOPTIONpeerreview
% \begin{center} \bfseries EDICS Category: 3-BBND \end{center}
% \fi
%
% For peerreview papers, this IEEEtran command inserts a page break and
% creates the second title. It will be ignored for other modes.
%\IEEEpeerreviewmaketitle




\item A die is loaded in such a way that each odd number is twice as likely to occur as each even number. Find P(G), where G is the event that a number greater than 3 occurs on a single roll of the die.\\
%\begin{table}[H]
	\centering
\begin{tabular}{|c|c|c|}
\hline
Random variable &Value &Definition\\ \hline
\multirow{3}{*}{X} &0 &Slips of Rs 1\\
&1 &Slips of Rs 5\\
&2 &Slips of Rs 13\\ \hline
\multirow{2}{*}{Y} &0 &Box A\\
&1 &Box B\\\hline
\end{tabular}
\caption{}
\label{tab:Distribution}
\end{table}
See \tabref{tab:Distribution}.
\begin{align}
p_{Y}\brak{k}= \begin{cases} 
      \frac{1}{3} & {k=0} \\
      \frac{2}{3 }& {k=1} 
   \end{cases}
   \\
p_{Y|X}\brak{0|0} = \frac{19}{25}\, 
p_{Y|X}\brak{0|1} = \frac{6}{25}\,
p_{Y|X}\brak{1|0} = \frac{45}{50}\,
p_{Y|X}\brak{1|2} = \frac{5}{50}
\end{align}
The desired probability is the probability that a slip drawn at random is marked other than Rs 1,
\begin{align}
&=1-p_X\brak{0}\\
&= p_X(1) + p_X(2)
\end{align}
Using Bayes theorem,
\begin{align}
&= p_Y\brak{0} \times \pr{Y=0 | X=1} + p_Y\brak{1} \times \pr{Y=1|X=2}\\
&=\frac{1}{3} \times \frac{6}{25} + \frac{2}{3} \times \frac{5}{50}\\
&=\frac{11}{75}
\end{align}

\newpage

%\tableofcontents

\bigskip

\renewcommand{\thefigure}{\theenumi}
\renewcommand{\thetable}{\theenumi}
%\renewcommand{\theequation}{\theenumi}

%\begin{abstract}
%%\boldmath
%In this letter, an algorithm for evaluating the exact analytical bit error rate  (BER)  for the piecewise linear (PL) combiner for  multiple relays is presented. Previous results were available only for upto three relays. The algorithm is unique in the sense that  the actual mathematical expressions, that are prohibitively large, need not be explicitly obtained. The diversity gain due to multiple relays is shown through plots of the analytical BER, well supported by simulations. 
%
%\end{abstract}
% IEEEtran.cls defaults to using nonbold math in the Abstract.
% This preserves the distinction between vectors and scalars. However,
% if the journal you are submitting to favors bold math in the abstract,
% then you can use LaTeX's standard command \boldmath at the very start
% of the abstract to achieve this. Many IEEE journals frown on math
% in the abstract anyway.

% Note that keywords are not normally used for peerreview papers.
%\begin{IEEEkeywords}
%Cooperative diversity, decode and forward, piecewise linear
%\end{IEEEkeywords}



% For peer review papers, you can put extra information on the cover
% page as needed:
% \ifCLASSOPTIONpeerreview
% \begin{center} \bfseries EDICS Category: 3-BBND \end{center}
% \fi
%
% For peerreview papers, this IEEEtran command inserts a page break and
% creates the second title. It will be ignored for other modes.
%\IEEEpeerreviewmaketitle





\item Determine the probability $p$,for each of following events.
\begin{enumerate}
\item An odd number appears in a single roll of dice.
\item Atleast one head appears in two tosses of fair coin.
\item A king,9 of hearts or 3 of spades appears in drawing a single card from a well shuffled deck of 52 cards.
\item The sum of 6 appears in single toss of a pair of fair dice.
\end{enumerate}
%\begin{table}[!ht]
\input{exemplar/12/13/3/41/tables/table.tex}
\caption{random variables of objects}
\label{tab:exemplar 12.13.3.41}
\end{table}
\begin{align}
\pr{X=i}=
\begin{cases}
\frac{1}{6},\text{when i=1}\\
\frac{2}{6},\text{when i=2}\\
\frac{3}{6},\text{when i=3}
\end{cases}
\end{align}

we know that that the conditional probability is defined as

                     $\pr{A|B}=\frac{\pr{A,B}}{\pr{B}}$


\begin{enumerate}
\item
The probability that a red ball will be selected is:
\begin{align}
\pr{Y=1}&=\pr{Y=1,X=1}+\pr{Y=1,X=2}+\pr{Y=1,X=3}\\
&=\pr{X=1}\times\pr{Y=1|X=1}+\pr{X=2}\times\pr{Y=1|X=2}+\pr{X=3}\times\pr{Y=1|X=3}\\
&=\frac{1}{6}\times\frac{3}{3}+\frac{2}{6}\times\frac{2}{3}+\frac{3}{6}\times0\\
&=\frac{7}{18}
\end{align}
\item
The probability that a white ball will be selected is:
\begin{align}
\pr{Y=0}&=\pr{Y=0,X=1}+\pr{Y=0,X=2}+\pr{Y=0,X=3}\\
&=\pr{X=1}\times\pr{Y=0|X=1}+\pr{X=2}\times\pr{Y=0|X=2}+\pr{X=3}\times\pr{Y=0|X=3}\\
&=\frac{1}{6}\times0+\frac{2}{6}\times\frac{1}{3}+\frac{3}{6}\times\frac{3}{3}\\
&=\frac{11}{18}
\end{align}
\end{enumerate}




\item Determine the probability p, for each of the following events.\\
(a) An odd number appears in a single toss of a fair die.\\
(b) At least one head appears in two tosses of a fair coin.\\
(c) A king, 9 of hearts, or 3 of spades appears in drawing a single card from a
well shuffled ordinary deck of 52 cards.\\
(d) The sum of 6 appears in a single toss of a pair of fair dice.
%\begin{table}[H]
	\centering
\begin{tabular}{|c|c|c|}
\hline
Random variable &Value &Definition\\ \hline
\multirow{3}{*}{X} &0 &Slips of Rs 1\\
&1 &Slips of Rs 5\\
&2 &Slips of Rs 13\\ \hline
\multirow{2}{*}{Y} &0 &Box A\\
&1 &Box B\\\hline
\end{tabular}
\caption{}
\label{tab:Distribution}
\end{table}
See \tabref{tab:Distribution}.
\begin{align}
p_{Y}\brak{k}= \begin{cases} 
      \frac{1}{3} & {k=0} \\
      \frac{2}{3 }& {k=1} 
   \end{cases}
   \\
p_{Y|X}\brak{0|0} = \frac{19}{25}\, 
p_{Y|X}\brak{0|1} = \frac{6}{25}\,
p_{Y|X}\brak{1|0} = \frac{45}{50}\,
p_{Y|X}\brak{1|2} = \frac{5}{50}
\end{align}
The desired probability is the probability that a slip drawn at random is marked other than Rs 1,
\begin{align}
&=1-p_X\brak{0}\\
&= p_X(1) + p_X(2)
\end{align}
Using Bayes theorem,
\begin{align}
&= p_Y\brak{0} \times \pr{Y=0 | X=1} + p_Y\brak{1} \times \pr{Y=1|X=2}\\
&=\frac{1}{3} \times \frac{6}{25} + \frac{2}{3} \times \frac{5}{50}\\
&=\frac{11}{75}
\end{align}

\newpage

%\tableofcontents

\bigskip

\renewcommand{\thefigure}{\theenumi}
\renewcommand{\thetable}{\theenumi}
%\renewcommand{\theequation}{\theenumi}

%\begin{abstract}
%%\boldmath
%In this letter, an algorithm for evaluating the exact analytical bit error rate  (BER)  for the piecewise linear (PL) combiner for  multiple relays is presented. Previous results were available only for upto three relays. The algorithm is unique in the sense that  the actual mathematical expressions, that are prohibitively large, need not be explicitly obtained. The diversity gain due to multiple relays is shown through plots of the analytical BER, well supported by simulations. 
%
%\end{abstract}
% IEEEtran.cls defaults to using nonbold math in the Abstract.
% This preserves the distinction between vectors and scalars. However,
% if the journal you are submitting to favors bold math in the abstract,
% then you can use LaTeX's standard command \boldmath at the very start
% of the abstract to achieve this. Many IEEE journals frown on math
% in the abstract anyway.

% Note that keywords are not normally used for peerreview papers.
%\begin{IEEEkeywords}
%Cooperative diversity, decode and forward, piecewise linear
%\end{IEEEkeywords}



% For peer review papers, you can put extra information on the cover
% page as needed:
% \ifCLASSOPTIONpeerreview
% \begin{center} \bfseries EDICS Category: 3-BBND \end{center}
% \fi
%
% For peerreview papers, this IEEEtran command inserts a page break and
% creates the second title. It will be ignored for other modes.
%\IEEEpeerreviewmaketitle




\item The probability distribution of a random variable X is given below:
\begin{table}[h]
    \centering
    \begin{tabular}{|c|c|c|c|c|}
        \hline
        X & 0 & 1 & 2 & 3 \\
        \hline
        $P(X)$ & $k$ & $\frac{k}{2}$ & $\frac{k}{4}$ & $\frac{k}{8}$ \\
        \hline
    \end{tabular}
\end{table}
\begin{enumerate}
    \item Determine the value of $k$.
    \item Determine $P(X \le 2)$ and $P(X > 2)$.
    \item Find $P(X \le 2)$ + $P(X > 2)$.
\end{enumerate}
\solution
%\begin{table}[H]
	\centering
\begin{tabular}{|c|c|c|}
\hline
Random variable &Value &Definition\\ \hline
\multirow{3}{*}{X} &0 &Slips of Rs 1\\
&1 &Slips of Rs 5\\
&2 &Slips of Rs 13\\ \hline
\multirow{2}{*}{Y} &0 &Box A\\
&1 &Box B\\\hline
\end{tabular}
\caption{}
\label{tab:Distribution}
\end{table}
See \tabref{tab:Distribution}.
\begin{align}
p_{Y}\brak{k}= \begin{cases} 
      \frac{1}{3} & {k=0} \\
      \frac{2}{3 }& {k=1} 
   \end{cases}
   \\
p_{Y|X}\brak{0|0} = \frac{19}{25}\, 
p_{Y|X}\brak{0|1} = \frac{6}{25}\,
p_{Y|X}\brak{1|0} = \frac{45}{50}\,
p_{Y|X}\brak{1|2} = \frac{5}{50}
\end{align}
The desired probability is the probability that a slip drawn at random is marked other than Rs 1,
\begin{align}
&=1-p_X\brak{0}\\
&= p_X(1) + p_X(2)
\end{align}
Using Bayes theorem,
\begin{align}
&= p_Y\brak{0} \times \pr{Y=0 | X=1} + p_Y\brak{1} \times \pr{Y=1|X=2}\\
&=\frac{1}{3} \times \frac{6}{25} + \frac{2}{3} \times \frac{5}{50}\\
&=\frac{11}{75}
\end{align}

\newpage

%\tableofcontents

\bigskip

\renewcommand{\thefigure}{\theenumi}
\renewcommand{\thetable}{\theenumi}
%\renewcommand{\theequation}{\theenumi}

%\begin{abstract}
%%\boldmath
%In this letter, an algorithm for evaluating the exact analytical bit error rate  (BER)  for the piecewise linear (PL) combiner for  multiple relays is presented. Previous results were available only for upto three relays. The algorithm is unique in the sense that  the actual mathematical expressions, that are prohibitively large, need not be explicitly obtained. The diversity gain due to multiple relays is shown through plots of the analytical BER, well supported by simulations. 
%
%\end{abstract}
% IEEEtran.cls defaults to using nonbold math in the Abstract.
% This preserves the distinction between vectors and scalars. However,
% if the journal you are submitting to favors bold math in the abstract,
% then you can use LaTeX's standard command \boldmath at the very start
% of the abstract to achieve this. Many IEEE journals frown on math
% in the abstract anyway.

% Note that keywords are not normally used for peerreview papers.
%\begin{IEEEkeywords}
%Cooperative diversity, decode and forward, piecewise linear
%\end{IEEEkeywords}



% For peer review papers, you can put extra information on the cover
% page as needed:
% \ifCLASSOPTIONpeerreview
% \begin{center} \bfseries EDICS Category: 3-BBND \end{center}
% \fi
%
% For peerreview papers, this IEEEtran command inserts a page break and
% creates the second title. It will be ignored for other modes.
%\IEEEpeerreviewmaketitle




\item %\begin{table}[H]
	\centering
\begin{tabular}{|c|c|c|}
\hline
Random variable &Value &Definition\\ \hline
\multirow{3}{*}{X} &0 &Slips of Rs 1\\
&1 &Slips of Rs 5\\
&2 &Slips of Rs 13\\ \hline
\multirow{2}{*}{Y} &0 &Box A\\
&1 &Box B\\\hline
\end{tabular}
\caption{}
\label{tab:Distribution}
\end{table}
See \tabref{tab:Distribution}.
\begin{align}
p_{Y}\brak{k}= \begin{cases} 
      \frac{1}{3} & {k=0} \\
      \frac{2}{3 }& {k=1} 
   \end{cases}
   \\
p_{Y|X}\brak{0|0} = \frac{19}{25}\, 
p_{Y|X}\brak{0|1} = \frac{6}{25}\,
p_{Y|X}\brak{1|0} = \frac{45}{50}\,
p_{Y|X}\brak{1|2} = \frac{5}{50}
\end{align}
The desired probability is the probability that a slip drawn at random is marked other than Rs 1,
\begin{align}
&=1-p_X\brak{0}\\
&= p_X(1) + p_X(2)
\end{align}
Using Bayes theorem,
\begin{align}
&= p_Y\brak{0} \times \pr{Y=0 | X=1} + p_Y\brak{1} \times \pr{Y=1|X=2}\\
&=\frac{1}{3} \times \frac{6}{25} + \frac{2}{3} \times \frac{5}{50}\\
&=\frac{11}{75}
\end{align}

\newpage

%\tableofcontents

\bigskip

\renewcommand{\thefigure}{\theenumi}
\renewcommand{\thetable}{\theenumi}
%\renewcommand{\theequation}{\theenumi}

%\begin{abstract}
%%\boldmath
%In this letter, an algorithm for evaluating the exact analytical bit error rate  (BER)  for the piecewise linear (PL) combiner for  multiple relays is presented. Previous results were available only for upto three relays. The algorithm is unique in the sense that  the actual mathematical expressions, that are prohibitively large, need not be explicitly obtained. The diversity gain due to multiple relays is shown through plots of the analytical BER, well supported by simulations. 
%
%\end{abstract}
% IEEEtran.cls defaults to using nonbold math in the Abstract.
% This preserves the distinction between vectors and scalars. However,
% if the journal you are submitting to favors bold math in the abstract,
% then you can use LaTeX's standard command \boldmath at the very start
% of the abstract to achieve this. Many IEEE journals frown on math
% in the abstract anyway.

% Note that keywords are not normally used for peerreview papers.
%\begin{IEEEkeywords}
%Cooperative diversity, decode and forward, piecewise linear
%\end{IEEEkeywords}



% For peer review papers, you can put extra information on the cover
% page as needed:
% \ifCLASSOPTIONpeerreview
% \begin{center} \bfseries EDICS Category: 3-BBND \end{center}
% \fi
%
% For peerreview papers, this IEEEtran command inserts a page break and
% creates the second title. It will be ignored for other modes.
%\IEEEpeerreviewmaketitle




\item Three persons, A, B and C, fire at a target in turn, starting with A. Their probability of hitting the target are 0.4, 0.3 and 0.2 respectively. The probability of two hits is\\
\begin{enumerate}
\item 0.024
\item 0.188
\item 0.336
\item 0.452
\end{enumerate}
\solution \\
%\begin{table}[H]
	\centering
\begin{tabular}{|c|c|c|}
\hline
Random variable &Value &Definition\\ \hline
\multirow{3}{*}{X} &0 &Slips of Rs 1\\
&1 &Slips of Rs 5\\
&2 &Slips of Rs 13\\ \hline
\multirow{2}{*}{Y} &0 &Box A\\
&1 &Box B\\\hline
\end{tabular}
\caption{}
\label{tab:Distribution}
\end{table}
See \tabref{tab:Distribution}.
\begin{align}
p_{Y}\brak{k}= \begin{cases} 
      \frac{1}{3} & {k=0} \\
      \frac{2}{3 }& {k=1} 
   \end{cases}
   \\
p_{Y|X}\brak{0|0} = \frac{19}{25}\, 
p_{Y|X}\brak{0|1} = \frac{6}{25}\,
p_{Y|X}\brak{1|0} = \frac{45}{50}\,
p_{Y|X}\brak{1|2} = \frac{5}{50}
\end{align}
The desired probability is the probability that a slip drawn at random is marked other than Rs 1,
\begin{align}
&=1-p_X\brak{0}\\
&= p_X(1) + p_X(2)
\end{align}
Using Bayes theorem,
\begin{align}
&= p_Y\brak{0} \times \pr{Y=0 | X=1} + p_Y\brak{1} \times \pr{Y=1|X=2}\\
&=\frac{1}{3} \times \frac{6}{25} + \frac{2}{3} \times \frac{5}{50}\\
&=\frac{11}{75}
\end{align}

\newpage

%\tableofcontents

\bigskip

\renewcommand{\thefigure}{\theenumi}
\renewcommand{\thetable}{\theenumi}
%\renewcommand{\theequation}{\theenumi}

%\begin{abstract}
%%\boldmath
%In this letter, an algorithm for evaluating the exact analytical bit error rate  (BER)  for the piecewise linear (PL) combiner for  multiple relays is presented. Previous results were available only for upto three relays. The algorithm is unique in the sense that  the actual mathematical expressions, that are prohibitively large, need not be explicitly obtained. The diversity gain due to multiple relays is shown through plots of the analytical BER, well supported by simulations. 
%
%\end{abstract}
% IEEEtran.cls defaults to using nonbold math in the Abstract.
% This preserves the distinction between vectors and scalars. However,
% if the journal you are submitting to favors bold math in the abstract,
% then you can use LaTeX's standard command \boldmath at the very start
% of the abstract to achieve this. Many IEEE journals frown on math
% in the abstract anyway.

% Note that keywords are not normally used for peerreview papers.
%\begin{IEEEkeywords}
%Cooperative diversity, decode and forward, piecewise linear
%\end{IEEEkeywords}



% For peer review papers, you can put extra information on the cover
% page as needed:
% \ifCLASSOPTIONpeerreview
% \begin{center} \bfseries EDICS Category: 3-BBND \end{center}
% \fi
%
% For peerreview papers, this IEEEtran command inserts a page break and
% creates the second title. It will be ignored for other modes.
%\IEEEpeerreviewmaketitle




item If two events are independent, then
\begin{enumerate}
\item they must be mutually exclusive
\item the sum of their probabilities must be equal to 1
\item (A) and (B) both are correct
\item None of the above is correct
\end{enumerate}
%Let $X$ be an bernoulli rv defined as in \tabref{tab:exemplar/11/16/3/26}.  Then, 
\begin{equation}
    p =
        \frac{4}{11} 
\end{equation}
\begin{table}[H]
	\centering
	\input{exemplar/11/16/3/26/tables/Table2.tex}
	\caption{}
        \label{tab:exemplar/11/16/3/26}
\end{table}

\item Three letters are dictated to three persons and an envelope is addressed to each of them, the letters are inserted into the envelopes at random so that each envelope contains exactly one letter. Find the probability that at least one letter in its proper envelope.\\
%\begin{table}[H]
	\centering
\begin{tabular}{|c|c|c|}
\hline
Random variable &Value &Definition\\ \hline
\multirow{3}{*}{X} &0 &Slips of Rs 1\\
&1 &Slips of Rs 5\\
&2 &Slips of Rs 13\\ \hline
\multirow{2}{*}{Y} &0 &Box A\\
&1 &Box B\\\hline
\end{tabular}
\caption{}
\label{tab:Distribution}
\end{table}
See \tabref{tab:Distribution}.
\begin{align}
p_{Y}\brak{k}= \begin{cases} 
      \frac{1}{3} & {k=0} \\
      \frac{2}{3 }& {k=1} 
   \end{cases}
   \\
p_{Y|X}\brak{0|0} = \frac{19}{25}\, 
p_{Y|X}\brak{0|1} = \frac{6}{25}\,
p_{Y|X}\brak{1|0} = \frac{45}{50}\,
p_{Y|X}\brak{1|2} = \frac{5}{50}
\end{align}
The desired probability is the probability that a slip drawn at random is marked other than Rs 1,
\begin{align}
&=1-p_X\brak{0}\\
&= p_X(1) + p_X(2)
\end{align}
Using Bayes theorem,
\begin{align}
&= p_Y\brak{0} \times \pr{Y=0 | X=1} + p_Y\brak{1} \times \pr{Y=1|X=2}\\
&=\frac{1}{3} \times \frac{6}{25} + \frac{2}{3} \times \frac{5}{50}\\
&=\frac{11}{75}
\end{align}

\newpage

%\tableofcontents

\bigskip

\renewcommand{\thefigure}{\theenumi}
\renewcommand{\thetable}{\theenumi}
%\renewcommand{\theequation}{\theenumi}

%\begin{abstract}
%%\boldmath
%In this letter, an algorithm for evaluating the exact analytical bit error rate  (BER)  for the piecewise linear (PL) combiner for  multiple relays is presented. Previous results were available only for upto three relays. The algorithm is unique in the sense that  the actual mathematical expressions, that are prohibitively large, need not be explicitly obtained. The diversity gain due to multiple relays is shown through plots of the analytical BER, well supported by simulations. 
%
%\end{abstract}
% IEEEtran.cls defaults to using nonbold math in the Abstract.
% This preserves the distinction between vectors and scalars. However,
% if the journal you are submitting to favors bold math in the abstract,
% then you can use LaTeX's standard command \boldmath at the very start
% of the abstract to achieve this. Many IEEE journals frown on math
% in the abstract anyway.

% Note that keywords are not normally used for peerreview papers.
%\begin{IEEEkeywords}
%Cooperative diversity, decode and forward, piecewise linear
%\end{IEEEkeywords}



% For peer review papers, you can put extra information on the cover
% page as needed:
% \ifCLASSOPTIONpeerreview
% \begin{center} \bfseries EDICS Category: 3-BBND \end{center}
% \fi
%
% For peerreview papers, this IEEEtran command inserts a page break and
% creates the second title. It will be ignored for other modes.
%\IEEEpeerreviewmaketitle




\end{enumerate}
