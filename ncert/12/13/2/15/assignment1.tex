\documentclass[12pt, journal]{IEEEtran}

\usepackage{tfrupee}
\usepackage{enumitem}
\usepackage{amsmath}
\usepackage{amssymb}
\usepackage{cite}
\usepackage{amsmath,amssymb,amsfonts,amsthm}
\usepackage{algorithmic}
\usepackage{graphicx}
\usepackage{textcomp}
\usepackage{xcolor}
\usepackage{txfonts}
\usepackage{listings}
\usepackage{enumitem}
\usepackage{mathtools}
\usepackage{gensymb}
\usepackage[breaklinks=true]{hyperref}
\usepackage{tkz-euclide} % loads  TikZ and tkz-base
\usepackage{listings}

\DeclareMathOperator*{\Res}{Res}
%\renewcommand{\baselinestretch}{2}
\renewcommand\thesection{\arabic{section}}
\renewcommand\thesubsection{\thesection.\arabic{subsection}}
\renewcommand\thesubsubsection{\thesubsection.\arabic{subsubsection}}

\renewcommand\thesectiondis{\arabic{section}}
\renewcommand\thesubsectiondis{\thesectiondis.\arabic{subsection}}
\renewcommand\thesubsubsectiondis{\thesubsectiondis.\arabic{subsubsection}}

% correct bad hyphenation here
\hyphenation{op-tical net-works semi-conduc-tor}
\def\inputGnumericTable{}                                 %%

\lstset{
%language=C,
frame=single, 
breaklines=true,
columns=fullflexible
}


\title{Assignment-1 \\ \Large AI1110: Probability and Random Variables \\ \large Indian Institute of Technology Hyderabad}
\author{Arugonda Srikar \\ \normalsize CS22BTECH11008}

\begin{document}
	\newtheorem{theorem}{Theorem}[section]
	\newtheorem{problem}{Problem}
	\newtheorem{proposition}{Proposition}[section]
	\newtheorem{lemma}{Lemma}[section]
	\newtheorem{corollary}[theorem]{Corollary}
	\newtheorem{example}{Example}[section]
	\newtheorem{definition}[problem]{Definition}
	%\newtheorem{thm}{Theorem}[section] 
	%\newtheorem{defn}[thm]{Definition}
	%\newtheorem{algorithm}{Algorithm}[section]
	%\newtheorem{cor}{Corollary}
	\newcommand{\BEQA}{\begin{eqnarray}}
	\newcommand{\EEQA}{\end{eqnarray}}
	\newcommand{\define}{\stackrel{\triangle}{=}}

	\bibliographystyle{IEEEtran}
	%\bibliographystyle{ieeetr}


	\providecommand{\mbf}{\mathbf}
	\providecommand{\pr}[1]{\ensuremath{\Pr\left(#1\right)}}
	\providecommand{\qfunc}[1]{\ensuremath{Q\left(#1\right)}}
	\providecommand{\sbrak}[1]{\ensuremath{{}\left[#1\right]}}
	\providecommand{\lsbrak}[1]{\ensuremath{{}\left[#1\right.}}
	\providecommand{\rsbrak}[1]{\ensuremath{{}\left.#1\right]}}
	\providecommand{\brak}[1]{\ensuremath{\left(#1\right)}}
	\providecommand{\lbrak}[1]{\ensuremath{\left(#1\right.}}
	\providecommand{\rbrak}[1]{\ensuremath{\left.#1\right)}}
	\providecommand{\cbrak}[1]{\ensuremath{\left\{#1\right\}}}
	\providecommand{\lcbrak}[1]{\ensuremath{\left\{#1\right.}}
	\providecommand{\rcbrak}[1]{\ensuremath{\left.#1\right\}}}
	\theoremstyle{remark}
	\newtheorem{rem}{Remark}
	\newcommand{\sgn}{\mathop{\mathrm{sgn}}}
	\providecommand{\abs}[1]{\left\vert#1\right\vert}
	\providecommand{\res}[1]{\Res\displaylimits_{#1}} 
	\providecommand{\norm}[1]{\left\lVert#1\right\rVert}
	%\providecommand{\norm}[1]{\lVert#1\rVert}
	\providecommand{\mtx}[1]{\mathbf{#1}}
	\providecommand{\mean}[1]{E\left[ #1 \right]}
	\providecommand{\fourier}{\overset{\mathcal{F}}{ \rightleftharpoons}}
	%\providecommand{\hilbert}{\overset{\mathcal{H}}{ \rightleftharpoons}}
	\providecommand{\system}{\overset{\mathcal{H}}{ \longleftrightarrow}}
	%\newcommand{\solution}[2]{\textbf{Solution:}{#1}}
	\newcommand{\solution}{\noindent \textbf{Solution: }}
	\newcommand{\cosec}{\,\text{cosec}\,}
	\providecommand{\dec}[2]{\ensuremath{\overset{#1}{\underset{#2}{\gtrless}}}}
	\newcommand{\myvec}[1]{\ensuremath{\begin{pmatrix}#1\end{pmatrix}}}
	\newcommand{\mydet}[1]{\ensuremath{\begin{vmatrix}#1\end{vmatrix}}}
	
		\maketitle
		\textbf{Question:12.13.2.15:}
		One card is drawn at random from a well shuffled deck of 52 cards. In which of
		the following cases are the events E and F independent ?
	\begin{enumerate}[label=(\roman*)]
		\item $E$ : ‘the card drawn is a spade’\\
			$F$ : ‘the card drawn is an ace’
		
		\item $E$: ‘the card drawn is black’\\
			$F$ : ‘the card drawn is a king’
		
		\item $E$ : ‘the card drawn is a king or queen’\\
			$F$ : ‘the card drawn is a queen or jack’.
	\end{enumerate}
	
	\textbf{Solution:}

	\begin{enumerate}[label=(\roman*)]
		\item 
			$E$ denotes the event that the card drawn is spade
			\begin{align}
				\pr{E} = \frac{13}{52} = \frac{1}{4} 
			\end{align}
			$F$ denotes the event that card drawn is ace 
			\begin{align}
				\pr{F} = \frac{4}{52} = \frac{1}{13}\\
				\pr{EF} = \frac{1}{52}\\
				\pr{E}\pr{F} = \frac{1}{4} \times \frac{1}{13} = \frac{1}{52}\\
				\therefore~\pr{EF} = \pr{E}\pr{F}
			\end{align}
			$\therefore$  $E$ and $F$ are independent events. \\
		\item
			$E$ denotes the event that the card drawn is black 
			\begin{align}
				\pr{E} = \frac{26}{52} = \frac{1}{2}
			\end{align}
			$F$ denotes the event that card drawn is a king 
			\begin{align}
				\pr{F} = \frac{4}{52} = \frac{1}{13} \\
				\pr{EF} = \frac{2}{52} = \frac{1}{26} \\
				\pr{E}\pr{F} = \frac{1}{2} \times \frac{1}{13} = \frac{1}{26}\\
				\therefore~\pr{EF} = \pr{E}\pr{F}
			\end{align}
			
			$\therefore$  $E$ and $F$ are independent events. \\
		\item
			$E$ denotes the event that the card drawn is king or queen
			\begin{align}
				\pr{E} = \frac{8}{52} = \frac{2}{13}
			\end{align}
			$F$ denotes the event that card drawn is a queen or jack 
			\begin{align}
				\pr{F} = \frac{8}{52} = \frac{2}{13}\\
				\pr{EF} = \frac{4}{52} = \frac{1}{13}\\
				\pr{E}\pr{F} = \frac{2}{13} \times \frac{2}{13} = \frac{4}{169}\\
				\therefore~\pr{EF} \neq \pr{E}\pr{F}
			\end{align}	
			$\therefore$  $E$ and $F$ are not independent events. \\
		
	\end{enumerate}
\end{document}
