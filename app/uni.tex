\begin{enumerate}[label=\thesubsection.\arabic*.,ref=\thesubsection.\theenumi]
\item  Let $X \in \cbrak{1,2,3,4,5,6}$ be the random variables representing the outcome for a die.  Assuming the die to be fair, the probability mass function (pmf) is expressed as 
\begin{align}
\label{eq:dice_pmf_xi}
p_{X}(n) = 
\begin{cases}
\frac{1}{6} & 1 \le n \le 6
\\
0 & otherwise
\end{cases}
\end{align}
\item The $Z$-transform of $X$ is given by 
\begin{align}
P_{X}(z) =  \frac{1}{6}\sum_{n = 1}^{6}z^{-n}
=\frac{z^{-1}\brak{1-z^{-6}}}{6\brak{1-z^{-1}}}, \quad \abs{z} > 1
\label{eq:dice_xiz}
\end{align}
upon summing up the geometric progression.  
\item From \eqref{eq:dice_xiz}, the CDF of $X$ is given by
\begin{align}
F_{X}\brak{n} = 	\pr{X \le n} 
 = 
\begin{cases}
0 & n < 1 \\
\frac{n}{6} & 1 \le n \leq 6 \\
1 & \text{otherwise}
\end{cases}
\label{eq:dice_xiF}
\end{align}
and plotted in \figref{fig:ncert/11/16/3/3/1}.
\begin{figure}[ht]
\centering
\includegraphics[width = \columnwidth]{ncert/11/16/3/3/figs/fig.png}
\caption{CDF}
\label{fig:ncert/11/16/3/3/1}
\end{figure}
\end{enumerate}
