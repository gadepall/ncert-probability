\documentclass[journal,12pt,two column]{IEEEtran}
%\usepackage{setspace}
\usepackage{listings}
\usepackage{amssymb}
\usepackage[cmex10]{amsmath}
\usepackage{amsthm}
\usepackage[export]{adjustbox}
\usepackage{bm}
\def\inputGnumericTable{} 

\usepackage[latin1]{inputenc}                                 
\usepackage{color}                                            
\usepackage{array} 
\usepackage{longtable} 
\usepackage{calc}                                             
\usepackage{multirow}                                         
\usepackage{hhline}                                           
\usepackage{ifthen}  
\usepackage{mathtools}
\usepackage{tikz}
\usepackage{listings}
\usepackage{color}                                            %%
\usepackage{array}                                            %%
\usepackage{caption} 
\usepackage{graphicx}

\title{A1110 Assignment 3 \\ 12.13.1.15}
\author{K.SaiTeja \\ AI22BTECH11014}
\providecommand{\pr}[1]{\ensuremath{\Pr\left(#1\right)}}
\providecommand{\qfunc}[1]{\ensuremath{Q\left(#1\right)}}
\providecommand{\sbrak}[1]{\ensuremath{{}\left[#1\right]}}
\providecommand{\lsbrak}[1]{\ensuremath{{}\left[#1\right.}}
\providecommand{\rsbrak}[1]{\ensuremath{{}\left.#1\right]}}
\providecommand{\brak}[1]{\ensuremath{\left(#1\right)}}
\providecommand{\lbrak}[1]{\ensuremath{\left(#1\right.}}
\providecommand{\rbrak}[1]{\ensuremath{\left.#1\right)}}
\providecommand{\cbrak}[1]{\ensuremath{\left\{#1\right\}}}
\providecommand{\lcbrak}[1]{\ensuremath{\left\{#1\right.}}
\providecommand{\rcbrak}[1]{\ensuremath{\left.#1\right\}}}
\newcommand*{\permcomb}[4][0mu]{{{}^{#3}\mkern#1#2_{#4}}}
\newcommand*{\perm}[1][-3mu]{\permcomb[#1]{P}}
\newcommand*{\comb}[1][-1mu]{\permcomb[#1]{C}}

\renewcommand{\thetable}{\arabic{table}} 
\newcommand{\question}{\noindent \textbf{Question: }}	
\newcommand{\solution}{\noindent \textbf{Solution: }}
\providecommand{\mbf}{\mathbf}
\providecommand{\pr}[1]{\ensuremath{\Pr\left(#1\right)}}
\providecommand{\prt}[2]{\ensuremath{P_{#1}^{\left(#2\right)} }}        % own macro for this question
\providecommand{\qfunc}[1]{\ensuremath{Q\left(#1\right)}}
\providecommand{\sbrak}[1]{\ensuremath{{}\left[#1\right]}}      % []
\providecommand{\lsbrak}[1]{\ensuremath{{}\left[#1\right.}}
\providecommand{\rsbrak}[1]{\ensuremath{{}\left.#1\right]}}
\providecommand{\brak}[1]{\ensuremath{\left(#1\right)}}         % ()
\providecommand{\lbrak}[1]{\ensuremath{\left(#1\right.}}
\providecommand{\rbrak}[1]{\ensuremath{\left.#1\right)}}
\providecommand{\cbrak}[1]{\ensuremath{\left\{#1\right\}}}      % {}
\providecommand{\lcbrak}[1]{\ensuremath{\left\{#1\right.}}
\providecommand{\rcbrak}[1]{\ensuremath{\left.#1\right\}}}
\theoremstyle{remark}
\newtheorem{rem}{Remark}
\newcommand{\sgn}{\mathop{\mathrm{sgn}}}
\providecommand{\abs}[1]{\ensuremath{\left\vert#1\right\vert}}
\providecommand{\res}[1]{\Res\displaylimits_{#1}} 
\providecommand{\norm}[1]{\ensuremath{\left\lVert#1\right\rVert}}
%\providecommand{\norm}[1]{\lVert#1\rVert}
\providecommand{\mtx}[1]{\mathbf{#1}}
\providecommand{\mean}[1]{\ensuremath{E\left[ #1 \right]}}
\providecommand{\fourier}{\overset{\mathcal{F}}{ \rightleftharpoons}}
%\providecommand{\hilbert}{\overset{\mathcal{H}}{ \rightleftharpoons}}
\providecommand{\system}{\overset{\mathcal{H}}{ \longleftrightarrow}}
	%\newcommand{\solution}[2]{\textbf{Solution:}{#1}}
\newcommand{\cosec}{\,\text{cosec}\,}
\providecommand{\dec}[2]{\ensuremath{\overset{#1}{\underset{#2}{\gtrless}}}}
\newcommand{\myvec}[1]{\ensuremath{\begin{pmatrix}#1\end{pmatrix}}}
\newcommand{\mydet}[1]{\ensuremath{\begin{vmatrix}#1\end{vmatrix}}}
%
%not used because document is short:
%\numberwithin{equation}{section}
%\numberwithin{figure}{section}
%\numberwithin{table}{section}
%\numberwithin{equation}{section}
%\numberwithin{problem}{section}
%\numberwithin{definition}{section}
\makeatletter
\@addtoreset{figure}{problem}
\makeatother

\let\StandardTheFigure\thefigure
\let\vec\mathbf
%\renewcommand{\thefigure}{\theproblem.\arabic{figure}}
    %\renewcommand{\thefigure}{\theproblem}
%\setlist[enumerate,1]{before=\renewcommand\theequation{\theenumi.\arabic{equation}}
%\counterwithin{equation}{enumi}
%\renewcommand{\theequation}{\arabic{subsection}.\arabic{equation}}

\def\putbox#1#2#3{\makebox[0in][l]{\makebox[#1][l]{}\raisebox{\baselineskip}[0in][0in]{\raisebox{#2}[0in][0in]{#3}}}}
     \def\rightbox#1{\makebox[0in][r]{#1}}
     \def\centbox#1{\makebox[0in]{#1}}
     \def\topbox#1{\raisebox{-\baselineskip}[0in][0in]{#1}}
     \def\midbox#1{\raisebox{-0.5\baselineskip}[0in][0in]{#1}}
\vspace{3cm}

%\renewcommand{\thefigure}{\theenumi}
%\renewcommand{\thetable}{\theenumi}
%\renewcommand{\theequation}{\theenumi}

\begin{document}
\maketitle
\question
Consider the experiment of throwing a die.
    \begin{itemize}
        \item If a multiple of 3 comes up, throw the die again
        \item If any other number comes, toss a coin.
    \end{itemize}
     Find the conditional probability of the event \lq the coin shows a tail\rq, given that \lq at least one die shows a 3\rq.\\
\solution\tableofcontents
\section{Markov Chain States}
\begin{itemize}
    \item  Let us construct a Markov chain $X_n$ with discrete time n.
    \item  The states $e_0$ and $e_1$ describe the outcomes from the latest dice throw.
    \item   The states $e_2$ and $e_3$ describe the outcomes of the latest coin toss.
\end{itemize}
\section{States}
Let $ Y \in \cbrak{1,2,3,4,5,6} $ denote the number obtained from a die throw. 
    \begin{table}[ht!]
        \centering
    	%%%%%%%%%%%%%%%%%%%%%%%%%%%%%%%%%%%%%%%%%%%%%%%%%%%%%%%%%%%%%%%%%%%%%%
%%                                                                  %%
%%  This is a LaTeX2e table fragment exported from Gnumeric.        %%
%%                                                                  %%
%%%%%%%%%%%%%%%%%%%%%%%%%%%%%%%%%%%%%%%%%%%%%%%%%%%%%%%%%%%%%%%%%%%%%%
\begin{tabular}{|c|l|l|}\hline
	\textbf{Variable}&\textbf{Description}&\textbf{Probability}\\\hline
A	&Person with heat attack	&\pr{A}=0.40\\\hline
$E_1$	&Person treated with meditation and yoga	&\pr{E_1}=0.50\\\hline
$E_2$	&Person treated with drug	&\pr{E_2}=0.50\\\hline
\end{tabular}

        \caption{States in Markov Chain}
        \label{table:States}	
    \end{table}
\section{Markov graph}
\begin{figure}[!ht]
        \centering
\begin{tikzpicture}[->, >= stealth, shorten >=2pt , line width =0.5 pt, node distance =2 cm]//
        \begin{tikzpicture}[->, >= stealth, shorten >=2pt , line width =0.5 pt, node distance =2 cm]//
\node [circle, draw] (0) at (0, 1.5) {$0$};
  \node [circle, draw] (1) at (3, 1.5) {$1$};
  \node [circle, draw] (2) at (5.5, 3) {$2$};
  \node [circle, draw] (3) at (5.5, 0) {$3$};
  
  \begin{small}
    \path (0) edge [loop left] node [left] {$p_{0/0} = \frac{2}{6}$} (0);
    \path (0) edge node [below = 0.2cm] {$p_{1/0} = \frac{4}{6}$} (1);
  
    \path (1) edge [bend left] node [above = 0.3cm] {$p_{2/1} = \frac{1}{2}$} (2);
    \path (1) edge [bend right] node [below = 0.3cm] {$p_{3/1} = \frac{1}{2}$} (3);
  
    \path (2) edge [loop right] node {$p_{2/2} = 1$} (2);
    \path (3) edge [loop right] node {$p_{3/3} = 1$} (3);
  \end{small}
        \end{tikzpicture}
  
        \end{tikzpicture}
        \caption{Graph of Markov Chain}
        \label{fig: markov_chain}
\end{figure}
\section{Description of Graphs and States}
 $p_{j/i}$ is the probability of moving from state $e_i$ to $e_j$.
    \begin{align}
    p_{j/i} = \pr{\frac{ X_{n+1}=j } {X_n=i} }
    \end{align}
    \textbf{Absorbing States}
    States $e_2$ and $e_3$ are absorbing states because $p_{2/2}=1$ and $p_{3/3}=1$. Once entered, they cannot be left.\\
    \textbf{Transient States}
    States $e_0$ and $e_1$ are transient states, because they lead to other states which have no return path, For example, $p_{2/1}=\frac{1}{2}$ but $p_{1/2}= 0$. Their probability will reduce to 0 eventually.   
\section{Transition Probability}
\textbf{State Probabilities in Next Step:}
{State Probabilities in Next Step}
    Let $\prt{i}{n}$ be the probability of state i at time n. 
    Then the state vector is,   
    \begin{align}
        \vec{Q_n} &= \myvec{\prt{0}{n}\\ 
        			        \prt{1}{n} \\
        			        \prt{2}{n} \\
        			        \prt{3}{n} 
        			        }        
    \end{align}    
    The probabilities after one step in time are
    \begin{align}
       \prt{0}{n+1} &= \frac{2}{6} \times \prt{0}{n}  \\
       \prt{1}{n+1} &= \frac{4}{6} \times \prt{0}{n}  \\
       \prt{2}{n+1} &= \frac{1}{2} \times \prt{1}{n} + 1 \times \prt{2}{n}
\\
       \prt{3}{n+1} &= \frac{1}{2} \times \prt{1}{n} + 1 \times \prt{3}{n} 
    \end{align}
    
\section{Transition Probability Matrix}

    The previous equations can be summarized as
    \begin{align}
        \vec{Q_{n+1}} &= \vec{P}\vec{Q_n} 
        \label{eq:transtition}        
    \end{align}
    Where $\vec{P}$ is the transition probability matrix. Its elements are values of $p_{i/j}$
    \begin{align}
        \vec{P}= \myvec{%%%%%%%%%%%%%%%%%%%%%%%%%%%%%%%%%%%%%%%%%%%%%%%%%%%%%%%%%%%%%%%%%%%%%%
%%                                                                  %%
%%  This is a LaTeX2e table fragment exported from Gnumeric.        %%
%%                                                                  %%
%%%%%%%%%%%%%%%%%%%%%%%%%%%%%%%%%%%%%%%%%%%%%%%%%%%%%%%%%%%%%%%%%%%%%%
\begin{tabular}{|l|l|}\hline
	Pr(Event)	&Value \\ \hline
	Pr($Y$=1 $\mid$ $X$=0) &0.25 \\ \hline
	Pr($Y$=1 $\mid$ $X$=1) &1  \\ \hline
	Pr($X$=0) &0.25 \\ \hline
	Pr($X$=1)	&0.75 \\ \hline
	
\end{tabular}
} 
    \end{align}

\section{Initial Condition and Limiting State}
\textbf{Initial Condition}
     The given condition is that \lq3 occurs at least once\rq. Let the first occurrence of 3 be the initial state $ \vec{Q_0}$.
    \begin{align}
        \vec{Q_0} &= \myvec{ 1 \\ 0 \\ 0 \\ 0 } 
    \end{align}
    Using equation \eqref{eq:transtition}, further states can be generated.
    \begin{align}
        \vec{Q_1} &= \vec{P} \vec{Q_{0}}
            = \myvec{\frac{2}{6} \\[4pt] \frac{4}{6}  \\[4pt] 0 \\ 0}\\
        \vec{Q_2} &= \vec{P} \vec{Q_1} = \vec{P}^{2} \vec{Q_0} 
            = \myvec{\frac{4}{9}  \\[4pt] \frac{8}{9}  \\[4pt] \frac{5}{24}\\[4pt] \frac{5}{12}} \\   
        \vdots \\
        \vec{Q_n} &= \vec{P}^{n} \vec{Q_0}
    \end{align}
\section{Eigen Decomposition}    
 Now to find the eigen values, let $\lambda$  be the eigen value for the transition matrix P\\
 \begin{align}
 \implies   \mid P-\lambda {I_4} \mid &= 0  
 \end{align}
 \begin{align}
 \myvec{%%%%%%%%%%%%%%%%%%%%%%%%%%%%%%%%%%%%%%%%%%%%%%%%%%%%%%%%%%%%%%%%%%%%%%
%%                                                                  %%
%%  This is a LaTeX2e table fragment exported from Gnumeric.        %%
%%                                                                  %%
%%%%%%%%%%%%%%%%%%%%%%%%%%%%%%%%%%%%%%%%%%%%%%%%%%%%%%%%%%%%%%%%%%%%%%

\begin{center}
    \begin{tabular}{|c|c|}
    \hline
	    \textbf{Digit}& \textbf{Favourable} \\ \hline
	    $X_{1}$ 		   & 	$5,7$	\\ \hline
	    $X_{2},X_{3},X_{4}$ 		   & 	$0,1,3,5,7$ \\ \hline

    \end{tabular}
    \end{center}
} &= 0
 \end{align}
 \begin{align}
 \implies (\frac{2}{6} - \lambda)(-\lambda)(1 - \lambda^2) = 0
 \end{align}
 \begin{align}
 \implies \lambda = \frac{2}{6}, 0 , 1 , 1 
 \end{align}
 \begin{center}    
$ \therefore $ eigen values\\
\end{center}
Now, finding the eigen vectors, say X\\
\begin{align}
\vec{X} &= \myvec{w \\ x \\ y \\ z} 
\end{align}
\begin{align}
\implies (P - \lambda I)X &= 0
\end{align}
\begin{enumerate}
\item $\lambda = \frac{2}{6}$
\begin{align}
\vec{X} &= \myvec{\frac{-2}{3}\\[4pt] \frac{-4}{3}\\[4pt] 1\\[4pt] 1}
\end{align}
\item $\lambda = 0$
\begin{align}
\vec{X} &= \myvec{0 \\[2pt] -2\\[2pt] 1\\ 1}
\end{align}
\item $\lambda = 1$
\begin{align}
\vec{X} &= \myvec{0 \\ 0 \\ 1 \\ 0} 
\myvec{0 \\ 0 \\ 0 \\ 1}\\
\end{align}
\end{enumerate}
Now, forming a eigenvector matrix\\
\begin{align}
\vec{S} &= \myvec{\input{tables/table5.txt}}
\end{align}
Now, we can write \\
\begin{align}
P &= SDS^{-1}
\end{align}
Where D is eigenvalue matrix,\\
\begin{align}
\vec{D} &= \myvec{\input{tables/table6.txt}}
\end{align}
To, find\\
\begin{align}
\lim_{n \to \infty} Q_{n} 
\end{align}
We know,
\begin{align}
Q_n &= P^{n}Q_0
\end{align}
and
\begin{align}
P^{n} &= (SDS^{-1})(SDS^{-1}) \dots (SDS^{-1})\\
\implies P^{n} &= SD^{n}S^{-1}\\
\implies \lim_{n \to \infty}P^{n} &= \lim_{n \to \infty}SD^{n}S^{-1}
\end{align}
Now,
\begin{align}
\vec{\lim_{n \to \infty}D^{n}} &= \myvec{\input{tables/table7.txt}}\\ 
\implies \vec{\lim_{n \to \infty}SD^{n}S^{-1}} &= \myvec{\input{tables/table7.txt}} = \vec{P^{n}}\\
\implies \vec{Q_n} &= \myvec{\input{tables/table7.txt}}\vec{Q_0}\\
\implies \vec{Q_n} &= \myvec{0 \\ 0 \\ 0 \\ 0} 
\end{align}
\textbf{Required Conditional Probability}
    Probability of the coin showing tails, given that at least one die shows a 3,
    \begin{align}
       \lim_{n \to \infty} \prt{3}{n} = 0
    \end{align}
    
\end{document}
