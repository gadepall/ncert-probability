\begin{enumerate}[label=\thesubsection.\arabic*,ref=\thesubsection.\theenumi]
\item The pdf of a Gaussian rv with mean $\mu$ and variance $\sigma^2$, defined as $Y \sim \gauss{\mu}{\sigma^2}$ is given by
	\begin{align}
	p_Y(x) = \frac{1}{\sqrt{2\pi} \sigma}e^{-\frac{\brak{x-\mu}^2}{2\sigma^2}}\quad x \in \brak{-\infty, \infty}.
\end{align}
\item Let
\begin{align}
	X \sim \bnm{n}{p}
\end{align}
The mean and variance are then given by 
\begin{align}
	\label{clt:bin-mu-sigma}
    \mu = np,\,
    \sigma^2 = npq.
\end{align}
\item For large $n$, 
\begin{align}
	Z = \frac{X-\mu}{\sigma} \system*{d}  \gauss{0}{1}
\end{align}
which implies that $Z$ converges in distribution to the standard Gaussian.  
%The standard Gaussian distribution has the probability density function (PDF)
%\begin{align}
%	p_Z(x) = \frac{1}{\sqrt{2\pi} \sigma}e^{-\frac{x^2}{2}} \quad x \in \brak{-\infty, \infty}.
%\end{align}
\\
\solution See
Appendix \ref{app:clt-binom}.
A comparison of the Binomial and Gaussian pmf/pdf is provided in 
	Figs. 
  \ref{fig:ncert/12/13/6/4/1}
  and
  \ref{fig:ncert/12/13/6/4/2}.
\begin{figure}[H]
	\centering
%\begin{subfigure}{0.4\textwidth}
  \includegraphics[width=\columnwidth]{ncert/12/13/6/4/figs/10.eps}
  \caption{10 trials}
  \label{fig:ncert/12/13/6/4/1}
\end{figure}
%\end{subfigure}
%\begin{subfigure}{0.4\textwidth}
\begin{figure}[H]
	\centering
  \includegraphics[width=\columnwidth]{ncert/12/13/6/4/figs/1000.eps}
  \caption{1000 trials}
  \label{fig:ncert/12/13/6/4/2}
%\end{subfigure}
\end{figure}
\item The CDF of $Z$ is
\begin{align}
F_{Z}\brak{x} &= \pr{Z \leq x}\\
&= \Phi_{Z}\brak{x}
\end{align}
\item The $Q$-function is defined as
\begin{align}
  \begin{aligned}
    \qfunc{x} &= \pr{Z > x}, \quad & x &> 0, \\
    \qfunc{-x} &= \pr{Z > -x}, \quad & x &< 0, \\
              &= 1 - \qfunc{x}. &
  \end{aligned}
\end{align}
\item 
\begin{align}
  \Phi_Z(x) =
	\begin{cases}
 1 - \qfunc{x}, &  x > 0, \\
 \qfunc{-x}, &  x < 0.
\end{cases}
\end{align}
%
\item 
\begin{align}
\label{clt:gauss-cdf-mu-sigma}
F_{Y}\brak{x} &= \begin{cases}
                 1 - \qfunc{\frac{x - \mu}{\sigma}},&  x > \mu\\
  \qfunc{\frac{\mu - x}{\sigma}},& x < \mu
               \end{cases}
\end{align}
\end{enumerate}
